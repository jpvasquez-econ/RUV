\documentclass[12pt]{article}
\usepackage[utf8]{inputenc}
\usepackage{tikz,egplot}
\usepackage[margin=0.8in]{geometry}
\usepackage{graphicx,ctable}
\usepackage{longtable}
\usepackage{hyperref,xcolor}
\hypersetup{colorlinks=false, linkbordercolor=red,pdfborderstyle={/S/U/W 1}}
\usepackage[para]{threeparttable}
\usepackage{tgpagella}
\usepackage[utf8]{inputenc}
\usepackage{natbib}
\usepackage{caption}
\usepackage{subcaption}
\usepackage{enumitem}
\usepackage[T1]{fontenc}
\setlength{\parindent}{0em}
\usepackage{forest}
\usepackage{booktabs,array,tabularx}
\usepackage{hyperref}
\newcommand{\tab}{\hspace*{3em}}
\hypersetup{
colorlinks = true,
urlcolor = blue,
}
\usepackage{setspace,amsfonts,amssymb,url,booktabs,tabularx,amsmath,amsthm,enumerate}

\title{ReadMe: DNWR and the China Shock}
\date{February 2023}

\begin{document}

\maketitle


\section*{General Description} 
We used ACS 1 year samples to replicate the ADH13
regression for the 2006-2020 period. For each year we used the reference year and the two neighbor years (e.g 2007 data is taken from pooled 2006-2008 ACS. We used ADH21 replication package code for presenting the graphs of coefficients through time. The outcome variables characterizing the labor market are simple and ten-year equivalent decadal changes of employment in manufacture, non-manuf employment, unemployment, and NILF. The outcomes are calculated as a share of working-age population. Following
ADH13, people living in institutional quarters and unpaid family members were dropped. We then created downward nominal wage rigidity measures to interact the effects of China's trade shock on unemployment.

\section{Raw data} 

\begin{itemize}
   \item \textbf{right2work.dta:} Right-to-work laws by state were taken from \href{https://nrtwc.org/facts/state-right-to-work-timeline-2016/}{here}.

    \item ACS data was taken from \href{https://usa.ipums.org/usa}{here}: We pooled
    the 2005-2021 ACS 1-year samples for people between 16 and 64 years
    of age. The name of this file is \textbf{ipums\_2005\_2021.dta} in the data
    file. A codebook file is available in  \textbf{ipums\_2005\_2021\_query.txt}. This document contains the query requested from the IPUMS USA webpage. A login is
    necessary to download the data. After an account is created, data can be downloaded from the ``Select Data'' window. The dataset was downloaded as a .dta
    for STATA.

   \item \textbf{workfile\_china.dta:} data taken from the replication package of
    ADH13, from the dta folder. This file contains the data to replicate the main         results of the paper.

   \item David Dorn's crosswalk files from PUMAs to CZ taken from his \href{https://www.ddorn.net/}{webpage}. The files
    are named \textbf{cw\_puma2000\_czone.dta} and \textbf{cw\_puma2010\_czone.dta.}

    \item \textbf{CPS 1986-1990:} data taken from IPUMS CPS \href{https://cps.ipums.org/cps/}{webpage}. The file name is \textit{cps\_86\_90.dta} The query if available in the file \textit{cps86\_90\_query.txt}. We followed Yoon Joo Jo (2021) on the data processing procedures.

    \item \textbf{Merged Outgoing Rotation Groups: 86-90:} this files are taken from \href{https://www.nber.org/research/data/current-population-survey-cps-merged-outgoing-rotation-group-earnings-data}{NBER} webpage, from the ``Stata dta files'' link. This files are merged to the CPS to obtain hourly wage allocation flag variable.

    \item \textbf{Jo\_state\_level\_dnwr.dta}: dataset provided by Ms. Yoon Joo Jo from her paper ``Establishing downward nominal wage rigidity through cyclical changes in the wage distribution'' (2022)

\end{itemize}

\section{Codes}

\begin{itemize}

    \item \textbf{1-ipums\_acs.do} This dofile takes as an input the pooled 2005-2021
    ACS 1-year samples, subsets 3-year samples, creates intermediate
    variables, and then merges the information to ADH2013's dataset workfile\_china.dta.

    \item \textbf{2-coef\_graphs\_decadal.do:} This code creates the outcome variables
    for manufacturing, non-manufacturing, nilf, and
    unemployment working population ratios. It then creates the coefficient graphs (figure 1 of technote1.pdf) for
    the period from 2006 to 2020. Outcome changes are in a ten-year equivalent form.

    \item \textbf{3-cps1990\_rigmeasures.do:} creates year-over-year wage change rigidity measures from the CPS 1986-1990 database following Joo-Jo,Y.(2022)

    \item \textbf{4-yjj\_rigidity\_measures.do:} creates year-over-year wage changes rigidity measures from the CPS database of Joo-Jo,Y.(2022). Variables were constructed using the period from 1997 to 2000.

    \item \textbf{5-dnwr\_figures.do:} creates and saves the figures for the rigidity measures coefficient graphs.

    \item \textbf{6-dnwr\_tables.do:} creates and saves the tables of the rigidity measures pdf file.
    
    \item \textbf{subfile\_ind1990dd:} This dofile was taken from David Dorn's data \href{https://www.ddorn.net/data.htm}{webpage}. It recodes the ind1990 variable into ind1990dd.
    This crosswalk code helps replicate the classification of 
    employment into manufacture and non-manufacture. 

\end{itemize}    


\section*{Initial Folder Structure}


\begin{forest}
  for tree={
    font=\ttfamily,
    grow'=0,
    child anchor=west,
    parent anchor=south,
    anchor=west,
    calign=first,
    edge path={
      \noexpand\path [draw, \forestoption{edge}]
      (!u.south west) +(7.5pt,0) |- node[fill,inner sep=1.25pt] {} (.child anchor)\forestoption{edge label};
    },
    before typesetting nodes={
      if n=1
        {insert before={[,phantom]}}
        {}
    },
    fit=band,
    before computing xy={l=15pt},
  }
[ADH
  [codes
    [All the codes detailed in section 2]
  ]
  [raw\_data
   [All \textit{.dta} and \textit{.txt} files detailed in section 1]
  ]
  [results
    [figures]
    [tables]
  ]
  [temp]
]



\end{forest}
    


\end{document}