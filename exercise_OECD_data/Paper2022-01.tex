\documentclass[12pt]{article}

\usepackage{amsmath,amssymb,amsthm,amsfonts,natbib,upgreek,color,adjustbox}
\usepackage{graphicx,bigstrut,float,enumerate,setspace,titlesec}
\usepackage{appendix,rotating,subcaption,enumitem,tikz,parskip}
\usepackage{comment,hyperref,siunitx,indentfirst,mathpazo,xcolor}
\usepackage[hang,flushmargin,bottom]{footmisc}
\usepackage[font=small,labelfont=bf,margin=1cm]{caption}
\usepackage[top=1.25in,bottom=1.25in,left=1.25in,right=1.25in]{geometry}

% Define some special colors
\definecolor{DarkGreen}{rgb}{0,0.4,0}
\definecolor{BlueBlind}{RGB}{0,114,178}
\definecolor{GreenBlind}{RGB}{0,158,115}
\definecolor{RedBlind}{RGB}{213,94,0}
\definecolor{DarkBlue}{rgb}{0,0,0.5}

% Code for aligning tables relative to the period
\sisetup{group-separator={\,}, table-align-text-pre=false, input-signs=+ -,
input-symbols={*} {**} {***}, input-open-uncertainty=, input-close-uncertainty=,}

% Some formatting
\hypersetup{colorlinks=true,linkcolor=DarkBlue,citecolor=DarkBlue,urlcolor=DarkBlue}
\allowdisplaybreaks
\setlength\parindent{2em}
\setlength{\parskip}{0em}
\renewcommand{\baselinestretch}{1.5}
\setlist{itemsep=2pt,topsep=0pt,parsep=0pt}

% Define some useful math notation
\def \e{{\epsilon}}
\def \ve{{\varepsilon}}
\def \vp{{\varphi}}
\def \b{{\beta}}
\def \d{{\delta}}
\def \D{{\Delta}}
\def \k{{\kappa}}
\newcommand{\be}{\small \begin{enumerate}}
\newcommand{\ee}{\end{enumerate}}
\newcommand{\bs}{\small \begin{eqnarray*}}
\newcommand{\es}{\end{eqnarray*}}
\newcommand{\bq}{\begin{eqnarray}}
\newcommand{\eq}{\end{eqnarray}}
\newcommand{\f}[2]{\frac{#1}{#2}}

% Title information
\title{{X}%
\thanks{XX}}
\author{X}
\date{\vspace*{0.5cm} January 2022}

\begin{document}

\maketitle

\thispagestyle{empty}

\begin{small}
\begin{spacing}{1.4}
\noindent

% Short abstract
\noindent
XX
\end{spacing}
\end{small}

\newpage

\pagenumbering{arabic}

\subsection{Data Description}\label{sec:data}

We use trade, production, and employment data for 50 U.S. states, 36 additional countries, and an aggregate rest of the world region, for a total of 87 regions from 2000 to 2007. We consider 14 sectors: 12 manufacturing sectors, one service sector, and one agricultural sector. All sectors are classified according to the North American Industry Classification System (NAICS). We provide a brief description of the data here and relegate additional details to Appendix \ref{sec:appendix_data}. 

{\textit{Labor, consumption, and input shares.}} For each region $j$ and each sector $k$, our model requires data to compute the share of labor in production $\phi_{j,k}$, the share of intermediates from all other sectors $\phi_{j,sk}\; \forall s$, and the aggregate consumption shares $\alpha_{j,k}$. We use data from the BEA (for U.S. states) and from WIOD to compute the share of value-added in gross output of region $j$, which in our model is equivalent to $\phi_{j,k}$. 
We also scale the relative importance of each U.S. state in the total value added of the U.S. so that the sum of value added across states matches the aggregate value-added of the U.S. according to WIOD. We compute $\phi_{j,sk}$ as the share of purchases of sector $k$ coming from sector $s$ (the input-output coefficient) using WIOD data.%
\footnote{We assume a common input-output matrix for all U.S. states, which is equal to the one of the U.S. as a whole.}  

{\textit{Bilateral trade flows.}} Our model also requires data on bilateral trade flows for all sectors and all regions in our sample in order to compute deficits, revenue, and trade shares for the year 2000. We also require the bilateral trade flows (combined with input-output coefficients) to infer the $\alpha_{j,k}$'s. We construct the bilateral trade flow dataset in four steps, which we summarize here while referring the reader to Appendix \ref{sec:appendix_data_bilat} for additional details. 

In the first step, we take sector-level bilateral trade between countries directly from WIOD. In the second step, we follow CDP to calculate the bilateral trade flows in manufacturing among U.S. states by combining WIOD and the Commodity Flow Survey (CFS). We first compute the bilateral expenditure shares across regions and sectors from the CFS, and then we use a proportionality rule to assign the total U.S. domestic sales from WIOD according to those bilateral shares. The bilateral trade flows matrix for the 50 U.S. states then match the total U.S. domestic sales from WIOD in each sector.

In the third step, we use the Import and Export Merchandise Trade Statistics, a dataset compiled by the U.S. Census Bureau, to compute -- for manufacturing and agriculture -- the sector-level bilateral trade flows between each U.S. state and each of the other countries in our sample. The U.S. Census data on exports at the sector-state-country level starts in 2002, and the data on imports starts in 2008. We use these starting years to project our bilateral trade matrix for previous years until 2000 by assuming that the importance of each state in the total exports (imports) to (from) other countries in each sector remains constant at the 2002 (2008) levels. We use a proportionality rule for the bilateral trade flows between the U.S. and other countries to match the values from WIOD in each sector.

In the fourth and last step, we combine data for region-level production and expenditure in services from the Regional Economic Accounts of BEA, WIOD data, and data on bilateral distances to construct the trade flows in services among all regions consistent with a gravity structure. We follow a similar gravity approach for the case of trade flows in agriculture using data from the Agriculture Census, the National Marine Fisheries Service Census, and WIOD. By construction, the bilateral trade flows in services and agriculture match the aggregates of trade in services and agriculture between all countries (including the U.S.) and the total production of U.S. services and agriculture consumed by the U.S. 

{\textit{Labor flows across sectors and regions.}} For the U.S. states, we construct the matrix of migration flows $\mu_{ji,sk,t}$ for $t = 2000$ combining data for intersectoral mobility from the Current Population Survey (CPS) with data for interstate mobility from the American Community Survey (ACS). We follow CDP by assuming that interstate movements ($j$ to $i$) across sectors follow the same pattern as the intrastate moves in the destination state $i$ across sectors. We then apply a proportionality rule to the flows from the CPS so that the total movements between states across sectors add up to the total movements in the ACS. An important limitation of measuring worker mobility across region-sectors using the self-reported information from the CPS and ACS is the well-known problem of artificially large amounts of mobility due to the prevalence of misclassification errors \citep{murphy1987unemployment,kambourov2008rising,kambourov2013cautionary,Dvorkin2021}. To avoid this issue, we smoothed the mobility flows in shares such that the set of migration flows in our first period implies a steady state in the U.S. in that period.\footnote{The change in the flows implied by this procedure is very small. In particular, the correlation between the original flows and the smoothed ones is 0.9969. We provide more details about our smoothing algorithm at the end of Appendix \ref{sec:appendix_data_migration}.} We then compute the initial allocation of labor in 1999 at the sector-region by combining the labor allocation in 2000 from the U.S. Census with data of mobility flows between 1999 and 2000. 

Finally, we assume away migration flows between countries. Thus, there is no need to compute labor flows for that case. In addition, for countries outside of the U.S., we assume that there are no costs of moving across sectors in each the single-region country due to data limitations. Given this assumption, one can infer the matrix of migration flows from the labor distribution in 1999 and 2000. We provide more details in Appendix \ref{sec:appendix_data_migration}.

\clearpage
\setcounter{page}{1}
\numberwithin{table}{section}
\numberwithin{figure}{section}
\numberwithin{equation}{section}
\renewcommand{\theequation}{\Alph{section}\arabic{equation}}
\setcounter{equation}{0}
\noindent
{\LARGE \textbf{Appendices for Online Publication}}
\setlength{\belowdisplayskip}{3pt}
\setlength{\belowdisplayshortskip}{3pt}
\setlength{\abovedisplayskip}{3pt}
\setlength{\abovedisplayshortskip}{3pt}

\let\normalsize\small
\appendix
\small

\section{Data Construction} \label{sec:appendix_data}

In this appendix section, we provide details on the construction of the data we briefly described in Section \ref{sec:data}. We divide this appendix into three parts. Appendix \ref{sec:appendix_data_sources} describes all data sources. Appendix \ref{sec:appendix_data_bilat} discusses how we combine the different data sources to compute an internally consistent bilateral trade-flow matrix for all sectors for the years when all the data is available. It also discusses how we use the previous step to project a bilateral trade-flows between states and countries for the years before full data availability. Finally, Appendix \ref{sec:appendix_data_migration} discusses the construction of the initial employment allocations for all regions and the  bilateral migration flows between sectors and U.S. states.

\subsection{Data Description and Sources}\label{sec:appendix_data_sources}

\textbf{List of sectors.} We use a total of 14 sectors. The list includes 12 manufacturing sectors, one catch-all services sector, and one agriculture sector. We follow CDP in the selection of the 12 manufacturing sectors. These are: \textbf{1)} Food, beverage, and tobacco products (NAICS 311-312, WIOD sector 3); \textbf{2)} Textile, textile product mills, apparel, leather, and allied products (NAICS 313-316, WIOD sectors 4-5); \textbf{3)} Wood products, paper, printing, and related support activities (NAICS 321-323, WIOD sectors 6-7); \textbf{4)} Mining, petroleum and coal products (NAICS 211-213, 324, WIOD sectors 2, 8); \textbf{5)} Chemical (NAICS 325, WIOD sector 9); \textbf{6)} Plastics and rubber products (NAICS 326, WIOD sector 10); \textbf{7)} Nonmetallic mineral products (NAICS 327, WIOD sector 11); \textbf{8)} Primary metal and fabricated metal products (NAICS 331-332, WIOD sector 12); \textbf{9)} Machinery (NAICS 333, WIOD sector 13); \textbf{10)} Computer and electronic products, and electrical equipment and appliance (NAICS 334-335, WIOD sector 14); \textbf{11)} Transportation equipment (NAICS 336, WIOD sector 15); \textbf{12)} Furniture and related products, and miscellaneous manufacturing (NAICS 337- 339, WIOD sector 16). There is a \textbf{13)} Services sector which includes Construction (NAICS 23, WIOD sector 18); Wholesale and retail trade sectors (NAICS 42-45, WIOD sectors 19-21); Accommodation and Food Services (NAICS 721-722, WIOD sector 22); transport services (NAICS 481-488, WIOD sectors 23-26); Information Services (NAICS 511-518, WIOD sector 27); Finance and Insurance (NAICS 521-525, WIOD sector 28); Real Estate (NAICS 531-533, WIOD sectors 29-30); Education (NAICS 61, WIOD sector 32); Health Care (NAICS 621-624, WIOD sector 33); and  Other Services (NAICS 493, 541, 55, 561, 562, 711-713, 811-814, WIOD sector 34).\footnote{The only difference with respect to CDP in the definition of manufacturing sectors is that we include Mining (NAICS 211-213) together with Petroleum and Coal Products (NAICS 324) in our sector 4.}

\textbf{List of countries:} We use data for 50 U.S. states, 37 other countries including a constructed rest of the world. The list of countries is: Australia, Austria, Belgium, Bulgaria, Brazil, Canada, China, Cyprus, the Czech Republic, Denmark, Estonia, Finland, France, Germany, Greece, Hungary, India, Indonesia, Italy, Ireland, Japan, Lithuania,
Mexico, the Netherlands, Poland, Portugal, Romania, Russia, Spain, the Slovak Republic, Slovenia, South Korea, Sweden, Taiwan, Turkey, the United Kingdom, and the rest of the world.

\textbf{Data on bilateral trade between countries.} World Input-Output Database (WIOD). Release of 2013. We use data for 2000-2007. We map the sectors in the WIOD database to our 14 sectors in the following way: \textbf{1)} Food Products, Beverage, and Tobacco Products (c3); \textbf{2)} Textile, Textile Product Mills, Apparel, Leather, and Allied Products (c4-c5); \textbf{3)} Wood Products, Paper, Printing, and Related Support Activities (c6-c7); \textbf{4)} Petroleum and Coal Products (c8); \textbf{5)} Chemical (c9); \textbf{6)} Plastics and Rubber Products (c10); \textbf{7)} Nonmetallic Mineral Products (c11); \textbf{8)} Primary Metal and Fabricated Metal Products (c12); \textbf{9)} Machinery (c13); \textbf{10)} Computer and Electronic Products, and Electrical Equipment and Appliances (c14); \textbf{11)} Transportation Equipment (c15); \textbf{12)} Furniture and Related Products, and Miscellaneous Manufacturing (c16); \textbf{13)} Construction (c18), Wholesale and Retail Trade (c19-c21), Transport Services (c23-c26), Information Services (c27), Finance and Insurance (c28), Real Estate (c29-
c30); Education (c32); Health Care (c33), Accommodation and Food Services (c22), and Other Services (c34); \textbf{14)} Agriculture and Mining (c1-c2). We follow \cite{Costinot2014} to remove the negative values in the trade data from WIOD.

\textbf{Data on bilateral trade in manufacturing between U.S states.} We combine the 2002 and 2007 Commodity Flow Survey (CFS) with the WIOD database. The CFS records shipments between U.S states for 43 commodities classified according to the Standard Classification of Transported Goods (SCTG). We follow CDP and use CFS 2007 tables that cross-tabulate establishments by their assigned NAICS codes against commodities (SCTG) shipped by establishments within each of the NAICS codes. These tables allow for mapping of SCTG to NAICS.

\textbf{Data on bilateral trade in manufacturing and agriculture between U.S states and the rest of the countries.} We obtain sector-level imports and exports  between the 50 U.S. states and the list of other countries from the Import and Export Merchandise Trade Statistics, which is compiled by the U.S. Census Bureau. This dataset  reports imports and exports in each NAICS sector between each U.S. state and each other country in the world. Data for exports at the  state$\times$sector level starts in 2002. Data for imports at the state$\times$sector level starts in 2008.

\textbf{Data on sectoral and regional value added share in gross output.} Value added for each of the 50 U.S. states and 14 sectors can be obtained from the Bureau of Economic Analysis (BEA) by subtracting taxes and subsidies from GDP data. In the cases when gross output was smaller than value added we constrain value added to be equal to gross output. For the list of other countries we obtain the share of value added in gross output using data on value added and gross output data from WIOD.

\textbf{Data on services expenditure and production.} We compute bilateral trade in services using a gravity approach explained in Appendix \ref{sec:appendix_data_bilat}. As part of this calculations we require data on production and expenditure in services by region. We obtain U.S. state-level services GDP from the Regional Economic Accounts of the Bureau of Economic Analysis (BEA). We obtain U.S. state-level services expenditure from the Personal Consumption Expenditures (PCE) database of BEA. Finally, for the list of other countries we compute total production and expenditure in services from WIOD.

\textbf{Data on agriculture expenditure and production.}  We also compute bilateral trade in agriculture using a gravity approach explained in Appendix \ref{sec:appendix_data_bilat}. To get production in agriculture for the U.S. states we combine the 2002 and 2007 Agriculture Census with the National Marine Fisheries Service Census to get state-level production data on crops and livestock and seafood. We infer state-level expenditure in agriculture from our gravity approach explained in Appendix \ref{sec:appendix_data_bilat}. Finally, for the list of other countries we compute total production and expenditure in agriculture from WIOD. 

\textbf{Data on population and geographic coordinates.} As part of the gravity approach to compute bilateral trade in services, we also need to compute bilateral distances between regions. We follow the procedure used in the GeoDist dataset of CEPII to calculate international (and intranational) bilateral trade distances. We thus require data on the most populated cities in each country, the cities' coordinates and population, and each country's population. We obtain this information from the United Nations' Population Division website. In particular, we use the population of urban agglomerations with 300,000 inhabitants or more in 2018, by country, for 2000-2007. For Austria, Cyprus, Denmark, Estonia, Hungary, Ireland, Lithuania, Slovakia and Slovenia we use the two most populated cities.\footnote{For the specific case of Cyprus, the cities' information comes from the country's Statistical Service.} For the case of U.S. states, we use population and coordinates data for each U.S county within each U.S state. The data for the U.S. counties comes from the U.S. CENSUS. 

\textbf{Data on employment and migration flows.} For the case of countries, we take data on employment by country and sector from the WIOD Socio Economic Accounts (WIOD-SEA). For the case of U.S. states, we take sector-level employment (including unemployment and non-participation) from the 5\% sample PUMS files of the 2000 Census. We only keep observations with age between 25 and 65, who are either employed, unemployed, or out of the labor force. We construct a matrix of migration flows between sectors and U.S. states by combining data from the American Community Survey (ACS) and the Current Population Survey (CPS). Finally, we abstract from international migration.

\subsection{Construction of the Bilateral Trade Flows Between Regions} \label{sec:appendix_data_bilat}

We follow the notation from \cite{Costinot2014} and omit the time subscripts $t$ that are relevant in our quantitative model. Define  $X_{ij,ks}$ as sales of intermediate goods from sector $k$ in region $i$ to sector $s$ in region $j$, and $X_{ij,kF}$ as the sales of sector $k$ in region $i$ to the final consumer of region $j$. Our final objective is to construct a bilateral trade flows matrix between all regions in our sample with elements equal to $X_{ij,k}=\sum_{s}X_{ij,ks}+X_{ij,kF}$. This matrix allows us to compute the trade shares $\lambda_{ij,k}$, and the sector-level revenues $R_{j,k}=\sum_{l}X_{jl,k}$ for each region, which are crucial elements in our hat algebra described in Section \ref{sec:hat_algebra}. 

As additional definitions, take $E_{j,k}=\sum_{i}X_{ij,k}$ as the total expenditure of region $j$ in sector $k$,  $F_{j,k}=\sum_{i}X_{ij,kF}$ as the final consumption in region $j$ of sector $k$, $F_{j}=\sum_{k}F_{j,k}$ as the total final consumption of region $j$, and $X_{j,ks}=\sum_{i}X_{ij,ks}$ as the total purchases that sector $s$ in region $j$ makes from sector $k$. We construct the matrix of $X_{ij,k}$ in four parts explained below. With some abuse of notation, we refer to a region $i$ as a U.S. state (country) by $i \in US$ ($i \notin US$). 

\textbf{Step 1: Bilateral trade between countries.} In the first step we focus on the case where both $i$ and $j$ are countries. Thus, we simply take $X_{ij,k}=X_{ij,k}^{WIOD}$, where $X_{ij,k}^{WIOD}$ are the bilateral trade flows that come directly from the WIOD database.

\textbf{Step 2: Manufacturing trade among U.S. states.} In the second step we focus on manufacturing bilateral trade between U.S. States. For this, we combine the closest Commodity Flow Survey (CFS) for each year with WIOD Data for the total trade of the U.S. with itself. We first compute the shares that each state $i$ exports to state $j$ in sector $k$ represent in the total trade of sector $k$ according to CFS. Then, we calculate the total exports of state $i$ to state $j$ in sector $k$ as WIOD's U.S. trade with itself in sector $k$ multiplied by the share computed in the previous step to ensure that bilateral trade between states adds up to the WIOD total.

\textbf{Step 3: Manufacturing trade between U.S. states and countries.} For the third step, we combine Census and WIOD data to calculate the trade flows between each of the 50 U.S. states and the other 37 country regions. There is limited availability for the state$\times$sector-level trade data coming from the  CENSUS. Data for exports at the state$\times$sector-level starts in 2002 and data for imports starts in 2008. We scale state-level imports and exports data from the Import and Export Merchandise Trade Statistics to match the U.S. totals in WIOD. More precisely, the exports (imports) of state $i$ to (from) country $j$ in manufacturing sector $k$ are computed as a proportion of WIOD's U.S. export (imports) to (from) country $j$ in sector $k$. This proportion is equal to the exports (imports) of state $i$ to (from) country $j$ in sector $k$ relative to the total U.S. exports (imports) to (from) country $j$ in sector $k$.

Since the Import and Export Merchandise Trade Statistics data for exports starts in 2002 and for imports starts in 2008, the bilateral trade flows between regions for the years before the data starts cannot be computed directly from the data. We adapt our computation method to take into account this issue. All previous procedures remain the same. Denote $X_{ij,k}^{base}$ as the matrix $X_{ij,k}$ for the first year where the exports or imports data is available (the base year). Define the share of exports of U.S. State $i$ in sector $k$, going to country $j$ in the base year as $y_{ij,k}^{base}\equiv\tfrac{X_{ij,k}^{base}}{\sum_{h\in US}X_{hj,k}^{base}} \quad \forall i\in US \, , \, j\notin US.$ Similarly, define the share of imports of U.S. state $j$ in sector $k$, coming from country $i$ in the base year as $e_{ij,k}^{base}\equiv\tfrac{X_{ij,k}^{base}}{\sum_{l\in US}X_{il,k}^{base}} \quad \forall i\notin US \, , \, j\in US.$ Finally for each sector $k$ in manufacturing or agriculture; and any year before the base year define $X_{ij,k}= e_{ij,k}^{base}X_{i\,US,k}^{WIOD} \quad \forall i\notin US, \; \forall j\in US$ and $X_{ij,k}=y_{ij,k}^{base}X_{US\,j,k}^{WIOD} \quad \forall i\in US, \; \forall j\notin US$.

\textbf{Step 4: Trade in services and trade in agriculture.} We compute bilateral trade flows for services and agriculture separately using a gravity structure that matches WIOD totals for trade between countries (including the U.S.). We start with the standard gravity equation (for simplicity, we remove the subscript of the sector) $X_{ij}=\left(\tfrac{w_{i}\tau_{ij}}{P_{j}}\right)^{-\varepsilon}E_{j},$
where $P_{j}^{-\varepsilon}=\sum_{i}\left(w_{i}\tau_{ij}\right)^{-\varepsilon}$. We know that $\sum_{j}X_{ij}=R_{i}$ and hence $\sum_{j}\left(\frac{w_{i}\tau_{ij}}{P_{j}} \right)^{-\varepsilon}E_{j}=R_{i}.$ This implies $w_{i}^{-\varepsilon} \Pi_{i}^{-\varepsilon}=R_{i}$, where $\Pi_{i}^{-\varepsilon}= \sum_{j} \tau_{ij}^{ -\varepsilon}P_{j}^{\varepsilon}E_{j}$. Let $\tilde{P}_{j}\equiv P_{j}^{-\varepsilon}$ and $\tilde{\Pi}_{i}\equiv \Pi_{i}^{-\varepsilon}$, and $\tilde{\tau}_{ij}\equiv\tau_{ij}^{-\varepsilon}$. Given $\left\{ E_{j}\right\}$,  $\left\{ R_{i}\right\}$,  and $\left\{ \tilde{\tau}_{ij}\right\}$, one we can get $\left\{ \tilde{P}_{j}\right\}$  and $\left\{ \tilde{\Pi}_{i}\right\}$ from the following system:
\begin{align}
\tilde{P}_{j}=&\sum_{i}\tilde{\tau}_{ij}\tilde{\Pi}_{i}^{-1}R_{i} \nonumber \\  \tilde{\Pi}_{i}=&\sum_{j}\tilde{\tau}_{ij}\tilde{P}_{j}^{-1}E_{j} \label{eq:grav_system_gen}
\end{align}
The solution for $\left\{ \tilde{P}_{j}, \tilde{\Pi}_{i} \right\}$ is unique up to a constant \citep{fally2015structural}. This indeterminacy requires a normalization. We thus impose $\tilde{P}_{1}=100$ in each exercise. Then one can compute our outcome of interest $\left\{ X_{ij}\right\}$  from
\bq
X_{ij}=\tilde{\tau}_{ij}\tilde{\Pi}_{i}^{-1}\tilde{P}_{j}^{-1}R_{i}E_{j}. \label{eq:grav_bilat}
\eq

\noindent \textit{{Computation of the bilateral resistance $\tilde{\tau}_{ij}$.}} To solve the gravity system, we must first compute $\tilde{\tau}_{ij}\; \forall i,\;j$. We proceed by assuming the following functional form: $\tilde{\tau}_{ij}=\beta_{0}^{\iota_{ij}}dist_{ij}^{\beta_{1}}\exp\left(\xi_{ij}\right),$ where $\iota_{ij}$ is an indicator variable equal to 1 if $i=j$, and $\xi_{ij}$ is an idiosyncratic error term. $\beta_{1}$ captures the standard distance elasticity and $\beta_{0}$ captures the additional \textit{inverse} resistance of trading with others versus with oneself. 

To calculate $dist_{ij}$, we follow the same procedure used in the GeoDist dataset of CEPII to calculate international (and intranational) bilateral trade distances. The idea is to calculate the distance between two countries based on bilateral distances between the largest cities of those two countries, those inter-city distances being weighted by the share of the city in the overall country's population \citep{head2002illusory}.

We use population for 2010 and coordinates data for all U.S. counties, and all cities around the world with more than 300,000 inhabitants. For those countries with less than two cities of this size, we take the two largest cities. Coordinates are important to calculate the physical bilateral distances in kms between each county $r$ in state $i$ and county $s$ in state $j$ ($d_{rs}\; \forall r\in i \; ,\; s\in j \text{ and }\;\forall i,j=1,...,50$), and define $dist\left(ij\right)$ as:
\begin{equation}
dist\left(ij\right) = \left(\sum_{r\, \in\, i} \sum_{s \, \in \, j} \left(\dfrac{pop_r}{pop_i}\right) \left(\dfrac{pop_s}{pop_j}\right) d_{rs}^\theta\right)^{1/ \theta},  \label{eq:distances}
\end{equation}
where $pop_h$ is the population of country/state $h$. We set $\theta=-1$.

Given our definition of $\tilde{\tau}_{ij}$ we can write the gravity equation between countries as $X_{ij}=\beta_{0}^{\iota_{ij}}dist_{ij}^{\beta_{1}}\exp\left(\xi_{ij}\right)\tilde{\Pi}_{i}^{-1}\tilde{P}_{j}^{-1}R_{i}E_{j}.$ Taking logs we can write the previous equation as: 
\begin{equation}
\ln X_{ij}=\delta_{i}^{o}+\delta_{j}^{d}+\tilde{\beta}_{0}\iota_{ij}+\beta_{1}\ln dist_{ij}+\xi_{ij}, \label{eq:gravity}
\end{equation}
where $\tilde{\beta}_{0}=\ln\beta_{0}$ and the $\delta$s are fixed effects. We first estimate the equation above separately for services and agriculture using a 2000-2011 panel of bilateral trade flows between countries from WIOD. We present our OLS estimation results in Table \ref{tab:owndummy}. Columns (1) and (2) refer to the estimated coefficients for the case of services and agriculture, respectively. Both regressions include year-by-origin and year-by-destination fixed effects. We take these estimates and compute the bilateral resistance term in each sector as $\hat{\tilde{\tau}}_{ij}=exp(\hat{\tilde{\beta}}_{0}\iota_{ij}+\hat{\beta}_{1}\ln dist_{ij})$. 

\begin{table}[ht] \caption{Estimation of Own-Country Dummy and Distance Elasticity}  
\begin{center}
{\def\sym#1{\ifmmode^{#1}\else\(^{#1}\)\fi}
\begin{tabular}{@{\extracolsep{5pt}}lccc@{}}
\hline\hline
  & (1)  & (2)  \\
Dep. Var.: $\ln X_{ij,t}$ & Services        & Agriculture      \\
\hline
$\iota_{ij}$              & 7.357 \sym{***} & 4.143\sym{***}   \\
                          & (0.126)         & (0.145)          \\
$\ln dist_{ij}$           & -0.376\sym{***} & -1.745\sym{***}  \\
                          &(0.037)          & (0.020)          \\
Year$\times$Orig.         &  Yes            &      Yes         \\
Year$\times$Dest.         &  Yes            &      Yes         \\
Observations              &    17,328       &    17,328        \\
Adjusted $R^2$            &     0.66        &     0.76         \\
\hline\hline
\end{tabular}}
\end{center}
\label{tab:owndummy} 
~~
\caption*{\footnotesize \textit{Notes:} This table displays the OLS estimates of specifications analogous to the one in equation \eqref{eq:gravity}. The outcome variable $\ln X_{ij,t}$ is the log exports of country $i$ sent to country $j$. The own-country dummy $\iota_{ij}$ is defined as an indicator function equal to one whenever country $i$ is the same as country $j$. Finally, $\ln dist_{ij}$ is the log distance between country $i$ and country $j$. This variable is computed according to equation \eqref{eq:distances}. Robust standard errors are presented in parenthesis. *** denotes statistical significance at the 1\%.\vspace{-0.5cm}}
\end{table}

\vspace{0.5cm}

\noindent \textit{{Trade in services.}} As inputs, we need total expenditures in services for each region $(E_i)$, as well as total production in services $(R_i)$. For the case of countries we take this directly from WIOD. For the case of U.S. states we take these variables from the Regional Economic Accounts of the Bureau of Economic Analysis. We scale the state-level services production and expenditures so that they aggregate to the U.S. totals in WIOD. 

We incorporate the information on bilateral trade in services between countries (including the U.S.) that comes from WIOD to the gravity system of equation \eqref{eq:grav_system_gen} by first writing the system as $\tilde{P}_{j}  =  \sum_{i\notin US}\tilde{\tau}_{ij}\tilde{\Pi}_{i}^{-1}R_{i}+\sum_{i\in US}\tilde{\tau}_{ij}\tilde{\Pi}_{i}^{-1}R_{i}$ and $\tilde{\Pi}_{i}  =  \sum_{j\notin US}\tilde{\tau}_{ij}\tilde{P}_{j}^{-1}E_{j}+\sum_{j\in US}\tilde{\tau}_{ij}\tilde{P}_{j}^{-1}E_{j}$. Then, we define $\tilde{\lambda}_{j}\equiv1-\tfrac{\sum_{i\notin US}X_{ij}}{E_{j}}$ for $j\notin US$ (the share of imports of region $j\notin US$ coming from the U.S.) and $\tilde{\lambda}_{i}^{*}\equiv 1-\tfrac{\sum_{j\notin US}X_{ij}}{R_{i}}$ for $i\notin US$ (total exports of region $i\notin US$ to other regions not in the U.S.). Using these two definitions and substituting  $\tilde{\tau}_{ij}=X_{ij}\tilde{\Pi}_{i}\tilde{P}_{j}R_{i}^{-1}E_{j}^{-1}$ whenever $i,j \notin US$ in the previous system of equations we have the final system we solve for services:  
\bs
\tilde{P}_{j} & = & \sum_{i}\tilde{\tau}_{ij}\tilde{\Pi}_{i}^{-1}R_{i}\quad j\in US \\
\tilde{\Pi}_{i} & = &\sum_{j}\tilde{\tau}_{ij}\tilde{P}_{j}^{-1}E_{j}\quad i\in US \\
\tilde{\lambda}_{j}\tilde{P}_{j} & = & \sum_{i\in US}\tilde{\tau}_{ij}\tilde{\Pi}_{i}^{-1}R_{i}\quad j\notin US \\
\tilde{\lambda}_{i}^{*}\tilde{\Pi}_{i} & = & \sum_{j\in US}\tilde{\tau}_{ij}\tilde{P}_{j}^{-1}E_{j}\quad i\notin US
\es

Once we find solutions for  $\left\{ \tilde{P}_{j}, \tilde{\Pi}_{i} \right\}$, we compute the final bilateral trade matrix according to equation \eqref{eq:grav_bilat}.

\noindent \textit{{Trade in agriculture.}} As inputs, we need total expenditures in services for each region $(E_i)$, as well as total production in agriculture $(R_i)$. For the case of countries we take this directly from WIOD. For the case of U.S. states we compute total production $(R_i)$ by combining data from the Agriculture Census and the National Marine Fisheries Service Census. We scale the state-level agriculture production so that it aggregates to the U.S. total in WIOD. However, it is not possible to find state-level agriculture expenditure for U.S. states. To overcome this data unavailability, we combine the U.S. input-output matrix $(\phi_{j,ks})$ together with the shares of value-added in gross production $(\phi_{j,k})$ in order to compute a value of $(E_i)$ that is consistent with the full bilateral trade matrix for all regions and all sectors. 

In order to describe our procedure, note that the total expenditure of region $j$ in sector $k$ $(E_{j,k})$ could be written as  $E_{j,k}=\sum_{s}\tilde{\phi}_{j,ks}R_{j,s}+F_{j,k},$ where $\tilde{\phi}_{j,ks}=\phi_{j,ks}(1-\phi_{j,s})$. We make two assumptions. First, we assume that $\tilde{\phi}_{j,ks}=\tilde{\phi}_{US,ks}$ $\forall j\in US$, which means that we assume common input-output matrix and value-added shares across U.S. states and equal to the ones of the U.S. as a whole. Second, we assume identical Cobb-Douglas preferences across U.S. states. This means that when $j\in US$ we have that $F_{j,k}=\tfrac{F_{j}}{F_{US}} F_{US,k}=F_{j}\gamma_{k}$, where $\gamma_{k}\equiv\tfrac{F_{US,k}}{F_{US}}$. Using these two assumptions we get $F_{j}= E_{j,k}-\sum_{s}\tilde{\phi}_{j,ks}R_{j,s}+\sum_{r\neq k}\left(E_{j,r}-\sum_{s}\tilde{\phi}_{j,rs}R_{j,s}\right).$ Substituting the previous equation in the definition of $E_{j,k}$ for the agriculture sector ($k=AG$), and $j\in US$ we find $$E_{j,AG}=\sum_{s}\tilde{\phi}_{j,AG\,s}R_{j,s}+\dfrac{\gamma_{AG}}{1-\gamma_{AG}}\sum_{r\neq AG}\left(E_{j,r}-\sum_{s}\tilde{\phi}_{j,rs}R_{j,s}\right),$$ which can be computed using state-level production of all sectors and state-level expenditure data of all other sectors (excluding agriculture), combined with the U.S.-level input-output matrix, value-added shares, and sector-level consumption shares. 

Once we obtain the state-level expenditure values in agriculture, we can proceed with the gravity system in equation \eqref{eq:grav_system_gen}. As in the case of services, we incorporate the information on bilateral trade in agriculture between countries that comes from WIOD. We also incorporate the bilateral trade in agriculture between U.S. states and other countries coming from the Import and Export Merchandise Trade Statistics. Thus, we only need to focus on $\left\{ \tilde{P}_{j}\right\}_{j\in US}$  and $\left\{ \tilde{\Pi}_{i}\right\}_{i\in US}$.  Define $\chi^{*}_{i}=1-\sum_{j\notin US}\tfrac{X_{ij}}{R_{i}}$ for $i\in US$ (the share of sales of state $i$ that stay in the U.S.) and $\chi_{j}=1-\sum_{i\notin US}\tfrac{X_{ij}}{E_{j,k}}$ for $j\in US$ (the share of purchases of state $i$ that come from the U.S.). The final system we solve for agriculture becomes:  
\bs
\chi_{j}\tilde{P}_{j} & =\Sigma_{i\in US}\tilde{\tau}_{ij}\tilde{\Pi}_{i}^{-1}R_{i},\forall j\in US \\
\chi^{*}_{i}\tilde{\Pi}_{i} & =\Sigma_{j\in US}\tilde{\tau}_{ij}\tilde{P}_{j}^{-1}E_{j},\forall i\in US
\es

As before, once we find solutions for  $\left\{ \tilde{P}_{j}, \tilde{\Pi}_{i} \right\}$, we compute the  bilateral trade in agriculture between U.S. states according to equation \eqref{eq:grav_bilat}.

\subsection{Initial Employment Allocations for each Region and Bilateral Migration Flows between Sectors and U.S. States}\label{sec:appendix_data_migration}

This subsection explains how to compute the initial labor allocation and the bilateral labor mobility matrix. Most of the steps follow CDP. 

\textbf{Employment allocation in each region and sector.} For the case of countries outside of the U.S., we first compute the employment distribution by country-sector from the WIOD-SEA. We treat unemployed and out of labor force as an additional sector. The data for that sector combines WIOD-SEA's worker population and each country's labor force participation rate from World Bank data. Since SEA does not include the RoW directly and since the remaining countries in SEA are too few, we define RoW's employment such that its production to employment ratio equals the respective average ratio of the other 37 countries. This calculation is done separately for each sector. 

For the case of U.S. states, we calculate the employment level for each state and sector (including unemployment and non-participation) in the year 2000 from the 5 \% sample PUMS files of the 2000 Census. We only keep observations type "P" (persons) aged 25 to 65, who are either employed, unemployed, or out of the labor force. We take unemployment plus non-participation as a different sector. Finally, we apply proportionality so that the aggregate employment at the sector level coincides with the totals for the U.S. in WIOD. 

\textbf{Workers' mobility matrix for U.S. states.} Let \(L_{ji,sk}\) be the number of workers who move from state $j$
and sector $s$ to state $i$ and sector $k$ between two periods (we ignore the time subscript for simplicity). We want to compute the mobility matrix for the shares  \(\mu_{ji,sk}\), for each origin state \(j\), origin sector \(s\), destination state \(i\), and destination sector \(k\), with the shares defined as  \(\mu_{ji,sk}=\frac{L_{ji,sk}}{\sum_{i'}\sum_{k'}L_{ji',sk'}}\).
To do this we combine data from the American Community Survey (ACS) and the Current Population Survey (CPS) as explained below. 

The ACS provides details of workers' current employment status, sector, and state. It also asks the state in which respondents lived the prior year. However, this survey does not provide information regarding people's employment status and sector in the previous year. This means that we can construct from the ACS data \(L^{ACS}_{ji,\#k}\:\forall j,i\in US\) and
destination sector $k$ (interstate flows but without knowing the sector of origin). The CPS  provides details of people's employment status and industry each month, but it does not provide information regarding movements across states. This means that we can construct from the CPS data \(L^{CPS}_{jj,sk}\:\forall j\in US\) and any origin or destination sectors $s,k$ (intra-state flows of people between sectors).\footnote{The CPS surveys households in a 4-8-4 format; that is, it interviews the household for four consecutive months, gives them an 8-month break, and interviews them again for four straight months. We match CPS observations (individuals) across time using the interview number. The first four monthly interviews are 12 months apart from the final four interviews, and the first four and final four are consecutive in months. Since we are interested in recording annual changes, we only keep interview months (1,5) which is equivalent to following each individual for the first twelve months she appears in the survey. To avoid noise in our sample, we use observations for the previous two years and the following two years for the year of interest.} 

To combine both $L^{ACS}_{ji,\#k}\:\forall j,i\in US$ and $L^{CPS}_{jj,sk} \:\forall j\in US$ to compute the labor transitions across states
and sectors, we follow CDP by assuming that interstates movements ($j$ to $i$) across sectors follow the same pattern that intrastate moves in the destination state $i$ across sectors. We then apply proportionality to the flows from CPS to sum up to total flows in ACS (which do not require additional assumptions and are available for interstate movements). This means that we define  $L_{ji,sk}=\sum_{k'}L^{ACS}_{ji,\#k'}\times\tfrac{L^{CPS}_{ii,sk}}{\sum_{s'}\sum_{k'}L^{CPS}_{ii,s'k'}} \quad \forall i,j\in US,\; \:\forall s,k.$ Note that \(\sum_{k}\sum_{s}L_{ji,sk}=\sum_{q}L^{ACS}_{ji,\#q}\) (so the
total movements between states add up to the total movements from ACS.). Also note that
$\frac{L_{jj,sk}}{L_{jj,sk'}}=\frac{L^{CPS}_{jj,sk}}{L^{CPS}_{jj,sk'}}$ (so that the relative importance between destination sectors comes from CPS data). Finally, in the few cases when the diagonal value of the matrix (same state and sector in origin and destination) is zero, we change it to the minimum non-zero diagonal value. 

\textbf{Smoothing flows in shares.} As discussed in the main text, using self-reported information from the CPS and ACS surveys to measure mobility flows is known to be problematic due to the prevalence of misclassification errors \citep{murphy1987unemployment,kambourov2008rising,kambourov2013cautionary,Dvorkin2021}. For instance, \cite{Dvorkin2021} shows that interindustry mobility rates computed using uncorrected PSID data could be around twice as large than alternative data for which misclassification is likely absent, even when using broad one-digit ISIC sector codes.  

To avoid the artificially large mobility flows due to the misclassification issue, we smoothed the mobility flows in shares such that the set of migration flows in our first period implies a steady state in the U.S. in that period. This smoothing means that given a set of $\mu_{ji,sk}$ coming from the data for 2000, we find a new set of flows $\mu'_{ji,sk}$ that satisfy the following conditions: 
\be
\item They are greater than zero $ \mu_{ji,sk}' \geq 0$ and they sum to one for each sender market over all receiver markets: $\sum_{i=1}^I \sum_{k=0}^S \mu'_{ji,sk} = 1$.
\item They imply a steady state with the labor data in 2000, which we will denote with $L_{j,s}$. This means that if the original distribution of labor is described by the $L_{j,s}$'s, this distribution is preserved after the flows occur: $L_{i,k}= \sum_{j=1}^I \sum_{s=0}^S \mu'_{ji,sk} L_{j,s}$.
\item The probability that someone in any given region-sector $is$ stays in region $i$ is the same across the original and the new mobility matrices $ \sum_{k=0}^S \mu_{ii,sk} = \sum_{k=0}^S \mu'_{ii,sk} $.
\item If the original mobility matrix has a given flow as zero, then this must still be the case in the new mobility matrix: $ \mu_{ji,sk}'=0 \; \text{ if } \; \mu_{ji,sk}=0 $.
\item The new $\mu_{ji,sk}'$ minimize the sum of square differences between the new $\mu$'s and the original ones, i.e.:
$$\sum_{j=1}^{I} \sum_{i=1}^{I} \sum_{s=0}^{S} \sum_{k=0}^{S} (\mu'_{ji,sk}-\mu_{ji,sk})^2.$$
\ee
We solve the previous problem of minimizing the sum of squared differences subject to the constraints in items 1-5. The change in the flows implied by this procedure is very small. In particular, the correlation between the original $\mu_{ji,sk}$ and the $\mu_{ji,sk}'$ is 99.69\%.

\textbf{Mobility matrix for non-U.S. regions.} We do not take the mobility matrix for each country outside of the U.S. from the data, which would be cumbersome because we have 37 other countries. However, it can be shown (details provided upon request), that for a country with a single region (such as non-U.S. countries in our context), the fact that there are no mobility costs can be captured by setting a special mobility matrix between 1999 and 2000. Thus, we compute the elements of that mobility matrix between 1999 and 2000. To do this, we take as given the labor distribution in 1999 ($L_{i,s,0}$) and 2000 ($L_{i,s,1}$) and compute the following formula:
\bs
\mu_{ii,sk,0} & = & \frac{L_{i,k,1}}{\sum_{r=1}^S L_{i,r,0}}
\es
Notice that the flows between sector $s$ and sector $k$ do not depend on information of the sender sector ($s$), which is implicitly encoding the information that in the countries outside of the U.S., mobility between sectors is unlimited.



\end{document}